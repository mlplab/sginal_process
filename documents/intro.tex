\chapter{はじめに}


信号処理技術とは,信号データの計測や解析,また解析するための技術を指す.
具体的には,音声や回路の電流値といったような計測器から計測されたアナログデータを
デジタルデータへサンプリング処理,
データに含まれている周波数成分の解析やデータの特徴となる周波数領域の抽出,
また計測時に生じたノイズを除去するためのフィルタ処理が挙げられ,
デジタルデータの前処理や音声の合成などに応用されている.
一方で,画像処理技術とは,画像における情報の解析,抽出や必要に応じた加工などのような画像データに対して
行われる処理技術の総称である.
具体的な例としては画像内のノイズ除去,輪郭の抽出,拡大や縮小,パターンマッチングなどが挙げられる.
これらの技術はレントゲン写真,CTスキャンの取得や解析といった医療現場,工場での製品の異常検知,
また撮影した画像に対しての編集といったような我々の身の回りの画像に対して幅広く活用されている.
最近では,コンピュータの発展に伴い機械学習技術を導入したより高度でかつ高精度な
画像処理技術が研究,開発されている.
信号処理技術には,データ内のノイズ除去や必要となる特徴の抽出などのように画像処理技術と
共通する技術がいくつか存在する.
このことから信号処理技術に用いられているフィルタ処理を画像処理に活用できるではないかと
考えた.

本レポートでは,信号処理技術における一部のフィルタ処理を画像データに用いることで
生じる利点や欠点,また画像処理技術との違いを実験,検証する.
